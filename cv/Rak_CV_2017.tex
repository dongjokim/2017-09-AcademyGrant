\documentclass[12pt]{article}
\usepackage {amsmath}
\usepackage {amssymb}
\usepackage {epsfig}
\usepackage {bm}
\usepackage {indentfirst} %indent the first par after section
\usepackage{setspace} 
\usepackage{color}
\usepackage{eurosym}

%\doublespacing
\oddsidemargin0cm
\topmargin-2cm     %I recommend adding these three lines to increase the 
\textwidth16.5cm   %amount of usable space on the page (and save trees)
\textheight23.5cm


\def\la{\left< }
\def\ra{\right> }
\def\mean#1{\ensuremath{\la#1\ra}}
\def\meanabs#1{\ensuremath{\la|#1|\ra}}
\def\meankv#1{\ensuremath{\la#1^2\ra}}
\def\rms#1{\meankv{#1}}
\def\sqrtrms#1{\ensuremath{\sqrt{\meankv{#1}}}}
\def\ptq#1{\ensuremath{\hat{p}_{\rm T#1}}} 
\def\ptqkv#1{\ensuremath{\hat{p}^2_{\rm T#1}}} 
\def\vptq#1{\ensuremath{\vec{\hat{p}}_{\rm T#1}}} 
\newcommand{\s} {\ensuremath{\sqrt{s}}}
\def\tev{\mbox{~TeV}}
\def\gev{\mbox{~GeV}}
\def\eg{{\it e.g.}}
\def\etc{{\it etc}}

\def\pt#1{\ensuremath{p_{\rm T#1}}} 
\def\ptkv#1{\ensuremath{p^2_{\rm T#1}}} 
\def\vpt#1{\ensuremath{\vec{p}_{\rm T#1}}} 

\def\kt#1{\ensuremath{k_{\rm T#1}}} 
\def\ktkv#1{\ensuremath{k^2_{\rm T#1}}} 
\def\vkt#1{\ensuremath{\vec{k}_{\rm T#1}}} 

\def\jt#1{\ensuremath{j_{\rm T#1}}} 
\def\jtkv#1{\ensuremath{j^2_{\rm T#1}}} 
\def\vjt#1{\ensuremath{\vec{j}_{\rm T#1}}} 

\newcommand{\mz} {\mean{z}}
\newcommand{\zt} {\ensuremath{z_{\rm t}}}
\newcommand{\mzt} {\mean{\zt}}
\newcommand{\za} {\ensuremath{z_{\rm a}}}
\newcommand{\mza} {\mean{\za}}
\newcommand{\xe} {\ensuremath{x_{\rm E}}}
\newcommand{\xh} {\ensuremath{x_{\rm h}}}
\newcommand{\xhq} {\ensuremath{\hat{x}_{\rm h}}}
\newcommand{\zkt} {\ensuremath{ \mean{\zt}\sqrtrms{\kt{}} }}
\newcommand{\xzkt} {\ensuremath{ \xhq^{-1}\mean{\zt}\sqrtrms{\kt{}} }}
\newcommand{\xzktfull} {\ensuremath{ \xhq^{-1}(\kt{},\xh)\mean{\zt(\kt{},\xh)}\sqrtrms{\kt{}} }}



\title{Curriculum Vitae}
\author{Jan Rak}
\date{\today}


\begin{document}

\maketitle

\parbox{7cm}{
Department of Physics\\
Survontie 9, Jyv\"askyl\"a\\
FI-40014 University of Jyv\"askyl\"a,\\ 
Finland}
\hfill
\parbox{5cm}{
Tel: +358 50 4280812\\
email: jan.rak@phys.jyu.fi\\
}

\vskip 2 cm

\noindent
\begin{tabular}{ll}
{\bf 1.1.2014} 		& Appointed as Professor in Experimental High Energy Nuclear Physics\\
				& at  Jyv\"askyl\"a University. Inauguration May 5. 2014. \\
{\bf 1.12.2006} 		& Docentship in Experimental High Energy Nuclear Physics at  Jyv\"askyl\"a University \\
{\bf 2005-today} 	& Research scientist, Jyv\"askyl\"a University \\
				& Member of ALICE collaboration.\\
				& Since 2007 ALICE/Finland project leader.\\
{\bf 2004-2005}  	& Research scientist, UNM Albuquerque, New Mexico, USA\\
				& Involved in the PHENIX spin physics program.\\
{\bf 2001-2004}  	& Research Scientist of Iowa State University, Ames, Iowa \\  						
				& at Brookhaven National Laboratory, Upton, New York, USA\\
				& Joined PHENIX collaboration. \\%Served as a data-production manager.\\
{\bf 1995-2000} 	& Postdoctoral research, Max Plank Institute Heidelberg\\
				& and GSI Darmstadt, Germany.\\ 
				& Member of CERES/NA45 collaboration. \\%Responsible for Si-drift detector system.\\
{\bf 19.1.1995} 		& Ph.D. defense:\\
				& ``The experimental study of the muon-capture by $^{28}Si$, $^{10}B$ and  $^{11}B$ nuclei.''\\
				& at Czech Technical University in Prague\\
				& Faculty of Nuclear Sciences and Physical Engineering \\
				& B\v{r}ehov\'a 7, 115 19 Prague 1, Czech Republic.\\
{\bf 1993-1995} 	& Ph.D. Nuclear Physics Institute, Academy of Science of Czech Republic.\\
{\bf 1989-1993} 	& Ph.D. Joint Institute for Nuclear Research, Dubna, Russia.\\
{\bf 1987-1989} 	& Junior Scientist, Faculty of Nuclear Sciences and Physical Engineering\\
				& Technical University in Prague, Czech Republic.\\
\\
{\bf Born} 	& February 9, 1962 in Prague, Czech Republic (Czech citizenship).\\
				
\end{tabular}


\clearpage

\section{Grants and funding:}
\begin{itemize}
\item 2011: received a 5-years grant ``Study of the deconfined QCD medium using high-pT probes with the ALICE experiment at
CERN LHC'' from the Research Council for Natural Sciences and Engineering, Finnish Academy 
(No. 251737) in amount of 540~k\euro. 

\item 2012 and 2015: one of the applicants (with Paula Erola, director of HIP, for CMS and Juha \"Ayst\"o, ) for the ``High-luminosity upgrades of the CMS and ALICE detectors at CERN'' (application 271838 and 293368). We received 455 k\euro\ in total (first year) for both CMS and ALICE upgrade projects. 

\item 2015-2018 Continuation of the previous FIRI grant. The funding we received was 507 k\euro.
\end{itemize}


\section{Professional Service:}
\begin{itemize}
\item Board member of the Particle Physics Division of the Finnish Physical Society.
\item Member of HIP detector laboratory advisory board. 
\item Chair of the ``International workshop on High-$p_{T}$ physics at LHC''.
\begin{itemize}
	\item 2006 Trento, Italy.
	\item 2007 Jyv\"askyl\"a, Finland.
	\item 2008 Tokay, Hungary.
	\item 2009 Prague, Czech Republic.
	\item 2010  Mexico City, Mexico.
	\item 2011 Utrecht, Netherlands.
	\item 2012 Frankfurt, Germany.
	\item 2012 Wuan China.
	\item 2013 Grenoble, France.
	\item 2014 Nantes, France.
	\item 2016 BNL, USA.
	\item 2017, Bergen, Norway.
\end{itemize}
\item Organizing committee of ÒTransverse Dynamics at RHICÓ, BNL workshop 2003
\item Organizer of ÒHigh-$p_{T}$Ó workshop at Annual RHIC\&AGS User Meeting, BNL 2004
\item International Advisory Committee of ÒPhase transitions in strongly interacting matterÓ conference, Prague 2004
\item Organizer of ÒHigh-$p_{T}$ and Jet PhysicsÓ at Annual RHIC\&AGS User Meeting, BNL 2005
\item Organizer of ÒJet Physcs sessionÓ at ISMD2005 conference, Kromeriz, Czech Republic 2005.
\item Organizer of ÒWorkshop on Critical Examination of RHIC ParadigmsÓ workshop in Austin, USA 2010
\item Member of the International Advisory Committee of ``Primordial QCD Matter in LHC Era, Implications of LHC Results on the Early Universe'' international conference.
\item Convener of heavy-ion session at 15th workshop on Elastic and Diffractive Scattering (EDS Blois workshop), Saariselka, Lapland, Finland.
\end{itemize}

\section{Editorial Services}
\begin{itemize}
\item Referee of Phys Rev Lett, European Phys Journal, Nuclear Physics A, J. Phys. G, IOP electronic journals.
\item Member of paper preparation group and the internal review committees for CERES, PHENIX and ALICE papers.

\end{itemize}

\section{Teaching}
\begin{itemize}
\item FYSH550: Experimental Ultra-relativistic Heavy Ion Physics (40h lectures and 20h exercises).
\item FYSH456: Experimental Methods in Particle Physics  (40h lectures and 20h exercises).
\item Lectures in Cracow School of Theoretical Physics, XLIV Course, 2004, New results in particle 
physics, Zakopane, Poland.
\item Lectures in Hyytialla summer school, 2006, Hyytialla, Finland.
\end{itemize}

\subsection{Students}
\begin{enumerate}
\item Oliver Nix, MPI, Germany, CERES experiment, 1996-2000, co-supervisor.
\item Jana Bielcikova, Uni. Heidelberg, CERES experiment, 1996-2000, co-supervisor.
\item Paul Constantin, BNL, USA, PHENIX experiment, 2000-2005, co-supervisor.
\item Nathan Grau, BNL, USA, PHENIX experiment, 2000-2005, co-supervisor.
\item Robert Hobbs,, BNL, USA, PHENIX experiment, 2000-2005, co-supervisor.
\item Rafael Diaz, PhD supervisor (left to industry before finishing PhD).
\item Norber Novitzky, Jyv\"askyl\"a University, PhD PHENIX experiment, supervisor (defense September 19th, 2013).
\item Ji\v{r}\'i Kr\'al, Jyv\"askyl\"a University, ALICE experiment, supervisor.
\item Beomsu Chang, Jyv\"askyl\"a University, ALICE experiment, PhD supervisor.
\item Jussi Viinikainen, suoervised MSc, now PhD supervisor ALICE.
\item Esko Pohjoisaho, Supervised MSc, now PhD supervisor ALICE.
\item Mikko Kervinen, Supervised MSc, now PhD supervisor ALICE.
\item Marton Vargyas, PhD supervisor.
\item Jasper Parkkila, PhD supervisor.
\item Tomas Snellman, MSc supervisor.
\item Petja Paakkinen, MSc supervisor.
\item Oskari Saarim\"aki, MSc supervisor.
\end{enumerate}

\subsection{Summer students}
2007 Timo Alho, 2008 Tiia Monto, 2009 Mikko Kervinen, Jussi Viinikainen, 2010 Subhashish Hazarica, 2011 Tiina Naaranoja, 2012	 Timo K\"arkk\"ainen, Shawana Hameed, Esko Pohjoisaho, Tomas Snellman. 2013 Petja Paakkinen,
2014 Juha Sulo, 2015 Henri H\"anninen, 2016 Jasper Parkkila,
2017 Oskari Saarim\"aki.


\section{Involvements in the WA98, CERES, PHENIX  and ALICE experiments}

\subsection{WA98 and CERES}

After finishing my Ph.D. I joined the heavy ion group at the Nuclear Physics Institute in Prague and became a member of WA98 collaboration, one of the heavy ion experiments on SPS accelerator at CERN
\footnote{CERES,WA98 - heavy ion experiments at Super Proton Synchrotron (SPS) accelerator at CERN, $\sqrt{s_{\rm NN}}$=17~GeV}. Our group was responsible for development and operation of  $Si$-drift detector tracking system \cite{Gatti:1984uu}. 

In 1995 I accepted a postdoctoral position in the group headed by professor J.P Wurm in Max Plank Institute in Heidelberg, Germany and joined the CERES experiment. In 1997-98  CERES  underwent a major upgrade and buit a large solenoidal Time Projection Chamber. During this period the Silicon vertex telescope consisting of two 3-inch Si-drift detectors was also upgraded. In collaboration with Pavel Rehak (BNL) and Peter Holl (MPI Munchen), we have designed a new ÒAZTEKÓ detector with larger acceptance (4-inch technology) and improved azimuthal resolution (interlaced anode structure).  The low input capacity ($\leq$~0.1~pF ) of this device required a development of  a new special front-end chip. We have developed a low-noise BiCMOS (0.8 $\mu$m) charge sensitive preamplifier ($\tau$=37ns) based on modification of the original design of  G.Gramegna, P.O'Connor, P.Rehak and S.Hart \cite{O'Connor:1998xv}.  I was also responsible for design, commissioning and operation of: 
\begin{itemize}
\item Active buffer system, which allowed the transfer of  analog signals from front-end chips to FADC system, and detector monitoring.   
\item Target zone mechanics allowing to operate the two silicon detectors in the limited space and high temperature environment.
\item First level centrality trigger electronics based on integration of $Si$-drift detector hit multiplicity.
\end{itemize}
All this hardware was implemented in the CERES experiment during the TPC upgrade and it was successfully operated in  1998, 1999 and 2000 runs.


\subsection{PHENIX at RHIC}

I joined the PHENIX\footnote{PHENIX - heavy ion exp. at Relativistic Heavy Ion Collider (RHIC) at BNL, USA, $\sqrt{s_{\rm NN}}$=200~GeV} collaboration  experiment  in May 2001 as an associated researcher in the group headed by Craig Ogilvie at Iowa State University. I began the high-\pt{}\  particle correlation analysis along similar lines as we did at SPS \cite{Agakichiev:2003gg}.  The main motivation of my research was to study the most intriguing results in the high-$p_T$ sector: the inclusive yield suppression, elliptic flow and di-jet correlations.  We performed a detailed shape analysis of the two-particle angular correlation function and developed new methods of extraction of the mean jet transverse momentum \sqrtrms{\jt{}}, mean intrinsic parton momentum \mean{\kt{}} and the fragmentation function deduced from the \xe\ distributions in $pp$, $dAu$ and $AuAu$ collisions. Main results of this study are summarized in \cite{Adler:2006sc}.

In the 2003 PHENIX run I served as the data production manager and I was responsible for online calibration and ÒonlineÓ data production.  We developed the framework allowing the filtering of the particular trigger sets. This helped significantly in online monitoring of the rare trigger efficiencies and we were able to monitor the $J/\Psi$ signal almost online. 

\subsection{ALICE at LHC}
Since 2007 I serve as a ALICE\footnote{ALICE - heavy ion exp. at Large Hadron Collider (LHC) at CERN, $\sqrt{s_{\rm NN}}$=2.76~TeV}/Finland
  project leader. My main responsibilities within the ALICE collaboration are listed below. 

\begin{itemize}
\item In 2008 I served as a period run coordinator and since January 2009 till April 2010  as a run coordinator for ALICE experiment. During my run-coordination term ALICE took the first $p+p$ at \s=900 \gev\ and later about 10 $nb^{-1}$ of \s=7 \tev\ data.
\item Member of collaboration board and participated in management board activities \eg\ formulation of the ALICE publication rules.
\item Chair of the paper preparation group of one of  "Measurement of the partonic transverse momentum in $p+p$ collisions at \s=0.9 and 7 \tev" utilizing the leading particle correlations. 
\item Member of the paper preparation committee of \cite{Aamodt2012}.
\item Convener of the high-\pt{} correlations working group (2010-2014).
\item Responsible for an ALICE EMCAL single photon trigger hardware and operations. Our group was involved in the hardware and firmware development of the EMCAL L0 trigger electronics \cite{JiriKral2012}.
\item Project leader of the GEM detector quality assurance project in Helsinki Institute of physics (2012-2018).
\end{itemize}



\section{Research Activities}
The major part of my scientific career has been on (p)QCD related phenomena and their use in heavy-ion physics.  I am interested in utilizing pQCD tools to study the partonic degree of freedom in heavy ion collisions, the low-x phenomena related to anticipated saturation of gluon and the spin structure functions, which is a subject of great importance for understanding the fundamental properties of QCD.  For the heavy ion program I am involved in the study of high-\pt{}\ jet/direct photon correlations with the goal to establish the modification of the parton properties induced by excited QCD medium.  The measurement of partonic primordial momenta, $k_T$, and the fragmentation function should shed a light on the process of parton interaction with the cold (p+A) and excited (A+A) QCD medium. 
The summary of the results related to my research could be found in the book ``High-$p_{T}$ physics in the Heavy Ion Era'' I wrote with Mike Tannenbaum  \cite{JanRak2013}.


\bibliographystyle{h-physrev3}
\bibliography{/Users/janrak/papers/bib-mypapers,/Users/janrak/papers/bib-ALICE,/Users/janrak/papers/bib-PHENIX}

\end{document}

 