%!TEX root = ResearchPlan_Rak.tex

\section{Research team and collaborative partners} %%%%%%%%%%%%%%%%%%%%%%%%%%%%%%%
\label{sec:reseachteam}

The Finnish involvement to ALICE experiment is carried out by the Department of Physics in University of Jyv\"askyl\"a (JYFL) and Helsinki Institute of Physics (HIP). At HIP the ALICE activities belong into the Nuclear Matter program coordinated by Ari Jokinen. Table~\ref{tab:personnel} shows the current members of ALICE analysis group in Jyv\"askyl\"a, the main persons involved in these activities, and their funding sources.

\begin{table}[htp]
\caption{ALICE-analysis group members in Jyv\"askyl\"a in September 2016.}
\begin{center}
\begin{tabular}{|l|l|l|l|l|}
\hline
   &  Name                       &  position         & Starting date & Funding \\
\hline
1 &    Jan  Rak                  & Professor       &  2005           & JYFL \\
2 &    DongJo   Kim          & Unicersity researcher     &  2006 & JYFL \\
3 &    Sami   R\"as\"anen & Unicersity researcher     &  2008 & JYFL \\
 \hline
4 &    Jussi   Viinikainen   & PhD student   & Jan-13         & HIP \\
5 &    Tomas Snellman     & PhD student   & June-13       & HIP \\
6 &    Marton Vargyas      & PhD student   & Jan-14         & JYFL \\
7 &    Jasper Parkkila       & PhD student   & Sep-17         & grant \\
\hline
8 &    Oskari Saarim\"aki & Undegraduate   &          &  \\
\hline
\end{tabular}
\end{center}
\label{tab:personnel}
\end{table}%

Related to this application, currently our group is involved with completing of flow and jet analysis with Run1 and Run2 data. From the PhD-students, Jussi Viinikainen and Marton Vargyas are working on jet analysis using two-particle correlations and Tomas Snellman studies fully reconstructed jets. Their topics are on hard physics. Jasper Parkkila does SC-correlators in flow analysis. Jasper finished his Master's Thesis in mid-September 2017 and has just started his PhD-thesis work. His thesis was graded as excellent (5/5) and he has already presented his analysis in ALICE Physics Working group.

Currently the post doc is not named, so this position would go into international call. Quite likely a suitable candidate could be a recently defended PhD inside the ALICE experiment, who has background in flow or EMCal jet analysis. He/she would then be very familiar with the experiment and analysis environment.

DongJo Kim is one of the flow experts in ALICE. He, for example, was the chair of the paper preparation committee related to recent ALICE publication on SC-correlators \cite{Acharya:2017gsw}. Sami R\"as\"anen has a theory background in hydrodynamical flow calculations brings understanding to the model comparisons.

Although CERN Physics Working Group (PWG) members are not directly involved with this project, they are providing invaluable support to the analysis work. Interaction of post doc and PhD-student with PWG will provide them substantial resource in guiding and making effective progress in the analysis work.

Besides the Jyv\"askyl\"a analysis group, we have two post doc's working in the HIP detector laboratory, Erik Brucken and Timo Hilden. Erik and Timo are working full time with the TPC upgrade. Other Finnish ALICE activities are pursued by senior researcher Wladyslaw Trzaska, who is leading the FIT project and professor Risto Orava who is leading ALICE-Forward group in HIP/Helsinki concentrating on diffractive physics.

\section{Research careers and researcher training}%%%%%%%%%%%%%%%%%%%%%%%%%%%%%%%
\label{sec:career}

CERN research guarantees mobility automatically to all post doc's and PhD-students. Successful working in large high energy experiment also gives an excellent starting point to progress in career since young researches have chances to make themselves known while they work in CERN, which is one more justification to spend significant period of time on-site. ALICE collaboration has over 1500 members from 154 institutions and 37 countries. This opens up numerous possibilities to continue their academic career.

\section{Mobility plan for the funding period}%%%%%%%%%%%%%%%%%%%%%%%%%%%%%%%
\label{sec:mobility}

Major upgrade to all LHC experiments, including ALICE, will take place during the long shutdown 2 (LS2) in 2019-2020. As discussed in Sec.~\ref{sec:researchmethods}, we are responsible EMCal level-0 (L0) trigger maintenance and operation and also participate in TPC upgrade and FIT-detector design, building and commissioning. FIT activities are included into separate project application to Academy by Wladyslaw Trzaska. 

Jasper Parkkila (PhD-student we will hire) will spend 1-2 years in CERN during the LS2 period. His main task will be the physics analysis related flow. This period is essential to attach to the large experimental community and build up personal network inside the experiment. Working on-site gives the best opportunity to make connections and become known inside the experiment. On top of that, all PhD-students in ALICE must contribute equivalent of 6 months of working time to do service work for the experiment. Jasper will participate in TPC upgrade activities during the time he will spend in CERN.

Our participation to TPC commissioning mainly involves installation and calibration work expected to take place during LS2 in years 2019-2020. Once the data taking starts again in 2021-2022, we need to ensure that temperature monitoring system works flawlessly. The post doc we will hire (N.N.) will use significant fraction of her working time to TPC upgrade over LS2, but the measurements with the data (cosmics and with first beams) will be duties of our next PhD-student, following Jasper Parkkila.

Once the beams come back, the duties of the post doc are more related to EMCal trigger maintenance. There is no major upgrade in the EMCal trigger that would be expected to change the software or hardware of the L0 trigger, particularly to the Trigger Region Units (TRU) that are on our responsibility. The main L0 trigger work start with the software commissioning tasks ensuring that the trigger levels are set and trigger validated, once the beams return to LHC. Our PhD-student Jussi Viinikainen performed this task during spring 2017 after EYETS (Extended Year End Technical Stop) and there is a clear documentation on all the procedures. Once Run 3 starts, we need guarantee running of the EMCal L0-trigger and provide regular trigger performance studies. Those goal are best met when there is an expert working on-site.

During LS2, some of the TRU boards needs to be fixed and spares produced, but those activities will be performed by electronic workshop at CERN or some private companies. As there are no trigger software activities expected over LS2, the tasks related to TPC upgrade and trigger maintanence do not overlap.

Both the post doc and PhD-student are expected to participate in international conferences and compete from opportunity to present their physics analysis results on behalf of ALICE. Annual travel money to this purpose is resourced to the project budget.

\nopagebreak
