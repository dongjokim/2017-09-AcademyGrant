%!TEX root = ResearchPlan_Rak.tex

\section{Research team and collaborative partners} %%%%%%%%%%%%%%%%%%%%%%%%%%%%%%%
\label{sec:reseachteam}

\textcolor{blue}{NOT AT ALL FINISHED.}

The Finnish involvement to ALICE experiment is carried out by the Department of Physics in University of Jyv\"askyl\"a (JYFL) and Helsinki Institute of Physics (HIP). At HIP the ALICE activities belong into the Nuclear Matter program coordinated by Ari Jokinen. Table~\ref{tab:personnel} shows the current members of ALICE analysis group in Jyv\"askyl\"a, the main persons involved in these activities, and their funding sources.

\begin{table}[htp]
\caption{ALICE-analysis group members in Jyv\"askyl\"a in September 2016.}
\begin{center}
\begin{tabular}{|l|l|l|l|l|}
\hline
   &  Name                       &  position         & Starting date & Funding \\
\hline
1 &    Jan  Rak                  & Professor       &  2005           & JYFL \\
2 &    DongJo   Kim          & Senior researcher     &  2006 & JYFL \\
3 &    Sami   R\"as\"anen & Senior researcher     &  2008 & JYFL \\
 \hline
4 &    Jussi   Viinikainen   & PhD student   & Jan-13         & HIP \\
5 &    Tomas Snellman     & PhD student   & June-13       & HIP \\
6 &    Marton Vargyas      & PhD student   & Jan-14         & JYFL \\
7 &    Jasper Parkkila       & PhD student   & Sep-17         & grant \\
\hline
8 &    Oskari Saarim\"aki & Undegraduate   &          &  \\
\hline
\end{tabular}
\end{center}
\label{tab:personnel}
\end{table}%

Related to this application, currently our group is involved with completing of flow and jet analysis with Run1 and Run2 data. Two of our PhD students Jussi Viinikainen and Marton Vargyas are finishing analysis of jet fragmentation in p-Pb and Pb--Pb collisions. In September 2016, J. Viinikainen  presented his results in the parallel talk at the Hard Probes conference in Wuan, China. The first paper on correlations between magnitudes of two different flow harmonics is just accepted in Phys. Rev. Lett.~\cite{ALICE:2016kpq} and the manuscript was selected a PRL Editors' Suggestion. DongJo Kim was one of the main contributors for this paper in ALICE and he is the chair of the 2$^{\rm nd}$ paper which will contain higher order flow harmonics correlations as well as the transverse momentum and rapidity dependence of the correlations. This 2nd paper is currently being prepared for collaboration review and these results will be presented in the "Quark Matter 2017" conference in February 2017. 



Besides the Jyv\"askyl\"a analysis group, we have two post doc's working in the HIP detector laboratory, Erik Brucken and Timo Hilden. Erik and Timo are working full time with the TPC upgrade. Other Finnish ALICE activities are pursued by senior researcher Wladyslaw Trzaska, who is leading the FIT project and professor Risto Orava who is leading ALICE-Forward group in HIP/Helsinki concentrating on diffractive physics.

Although CERN Physics Working Group (PWG) members are not directly involved with this project, they are providing invaluable support to the analysis work. Interaction of post doc and PhD-student with PWG will provide them substantial resource in guiding and making effective progress in the analysis work.

\section{Research careers and researcher training}%%%%%%%%%%%%%%%%%%%%%%%%%%%%%%%
\label{sec:career}

\textcolor{blue}{NOT AT ALL FINISHED.}

All the closest seniors in this project, the PI, DongJo Kim and Sami R\"as\"anen, clearly fulfill the Academy requirements for mobility. CERN research guarantees mobility automatically to all post doc's and PhD-students. Successful working in large high energy experiment also gives an excellent starting point to progress in career since young researches have chances to make themselves known while they work in CERN, which is one more justification to spend significant period of time on-site. ALICE collaboration has over 1500 members from 154 institutions and 37 countries. This opens up numerous possibilities to continue their academic career.

\section{Mobility plan for the funding period}%%%%%%%%%%%%%%%%%%%%%%%%%%%%%%%
\label{sec:mobility}

\textcolor{blue}{1st DRAFT TEXT.}

Major upgrade to all LHC experiments, including ALICE, will take place during the long shutdown 2 (LS2) in 2019-2020. As discussed in Sec.~\ref{sec:researchmethods}, we are responsible EMCal level-0 (L0) trigger maintenance and operation and also participate into TPC ROC upgrade and FIT-detector design, building and commissioning. FIT activities are included into separate project application to Academy by Wladyslaw Trzaska. The post doc we will hire should use a significant fraction of her time to fulfil our obligations related to TPC and EMCal activities.

The work related to TPC and EMCal should not overlap very significantly since TPC work will take place mainly during the LS2 and L0 trigger once the Run 3 starts in 2021. There is no major upgrade in the EMCal trigger that would be expected to change the software or hardware of the L0 trigger, namely to the Trigger Region Units (TRU). Some of the TRU boards needs to be fixed and spares produced, but those activities will be performed by electronic workshop at CERN or some private company. Post doc’s main L0 trigger work start with the software commissioning tasks, where the trigger levels are set and trigger validated, once the beams return to LHC. Our PhD-student Jussi Viinikainen performed this task during spring 2017 after EYETS (Extended Year End Technical Stop). Once the Run 3 starts, we need guarantee flawless running of the EMCal L0-trigger and provide regular trigger performance studies. This is the best achieved when the expert is on site.

Our participation to TPC commissioning mainly involves installation and calibration work expected to tako place during LS2 in years 2019-2020. One the data taking starts again in 2021-2022, we need to ensure that temperature monitoring system works flawlessly. 

\textcolor{blue}{OBSERVE: overlap in post doc's duties with EMCal duty.}

The student and PostDoc will participate in the preparation for the upgrade, in the pre-commissioning of the new readout chambers after installation onto the TPC field cage on the surface, in the commissioning of the TPC after installation in the experiment and in the calibration of the first data recorded with the detector in the years 2019 to 2022. 

Jasper Parkkila (PhD-student we will hire) will spend 1-2 years in CERN during the LS2 period. His main task will be in the physics analysis related flow. This period is essential to attach to the large experimental community and build up personal network inside the experiment. Working on-site gives the best opportunity to make connections and become known inside the experiment. On top of that, all PhD-students in ALICE must contribute equivalent of 6 months of working to time experiment. Jasper will participate to TPC upgrade activities during the time he will spend in CERN.

Both the post doc and PhD-student are expected to participate into international conferences and compete from opportunity to present their physics analysis results on behalf of ALICE. Annual travel money to this purpose is resourced to the project budget.

\nopagebreak
